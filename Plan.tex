\documentclass[UTF8]{dingo}

\author{康赣鹏 \\ DingoLab \\ganpengkang@gmail.com}
\definecolor{cb}{rgb}{1.00,0.50,0.50}
\definecolor{cf}{rgb}{1,1,1}
\title{Dingo 项目计划书}
\titlebgcolor{cb}
\titlefontcolor{cf}
\coverpic{Dingo-B}
\city{西安}
\version{0.1.0.0}
\begin{document}
  \makecover
  \makecontent
  \section{范围}
	  \subsection{标识}
		  本文档为pdf文件 适用于各个操作系统
	  \subsection{系统概述}
		  \begin{itemize}
		  	\item 文档适用的系统为(window mac linux等等)
		  	\item 软件用途:用户通过我们提供app软件客户端,可查看是否有人在快递集散地,并通过我们的软件协商实现交易, 消费掉这最后一公里。
		  	\item 软件特性:主要由后端和客户端组成,客户端在wp平台上开发,后端主要用Haskell语言开发 数据库开发软件为PsotgreSQL
		  	\item 软件尚处于开发阶段 开发方主要由4名学生组成
		  	\item 有关文档:产品需求文档、API文档、后端文档、前端文档
		  \end{itemize}
		\subsection{文档概述}
			本文档主要阐述了开发项目的背景,项目在国内外市场测分析,项目主要开发建设的内容和项目实施的技术方案。另外,对项目人员的组织,资源的分配以及客户的交流做了初步分析
		\subsection{术语定义}
		\begin{enumerate}
			\item API文档:确定前后端接口的文档
			\item 后端文档:项目后端设计文档
			\item 前端文档:项目前端设计文档
			\item 测试文档:测试程序的是否能全面完成功能的文档
			\item 产品需求文档:分析一个产品要实现多少功能的文档
		\end{enumerate}
  \section{引用文档及参考资料}
	  \begin{tabular}{|c|c|c|}
	  	\hline
	  	Haskell Wiki & Hackage Document & PostgreSQL官方手册 \\ \hline
	  	msdn开发者文档 & 知乎日报uwp版 & 博客园客户端uwp版 \\ \hline
	  \end{tabular}
  \section{策划背景概述}
    随着网购逐渐的渗透于我们的生活,快递也成为我们生活必不可少的一部分。快递方便快捷,为我们的 生活减轻了不少负担。
    然而在许多地区,因为各种各样的原因,快递还是不能实现把商品直达家庭的目标。
    存在着“最后一公里”,这与快递这种服务诞生的理念相悖。

    根据我的调查,在中国的大部分高校。因为快递的种类繁杂,人员众多,以及宿舍的集体生活,
    大部分 快递都选择在学校租赁一个快递集散地,让学生自取。这造成一种现象就是:每当下课铃响后,学校的
    两个地方就会挤满人,一个是饭堂,一个就是快递集散地。将宿舍和快递集散地这段路程利用起来非常 有必要。

    还有,在中国的大部分农村,手机通讯,⺴络的搭建已经开始普及了,但运输快递的⺴络雏形还没有出 现。
    大部分的村子,想要领取快递必须到附近的镇子上,耗时⻓还费劲。快递公司若要拓展这部分市 场,必须投入大量人力物力,
    还很难收回成本。这时候,我们可以利用的就是人民自身的力量,填补这 部分市场的空缺。

    在国外,许多地干人稀的地方通过类似于快递集散地的驿站,快递的最后一公里还是得客户去消化。
    这 证明,无论是在国内还是国外,“最后一公里”得问题还是没有得到有效得解决,这部分市场的空间很 大。

  \section{项目开发活动的总体实施计划}
    \subsection{软件开发过程}
      软件生命周期模型为敏捷开发的生命周期模型,软件开发方法为敏捷开发。
      开发思路按照:需求 $\rightarrow$ 分析  $\rightarrow$ 设计  $\rightarrow$ 编码  $\rightarrow$ 测试的基本思路进行,
      并在此基础上做出改进。将架构设计细分成三个部分,针对每个部分进行详细设计 $\rightarrow$ 编码 $\rightarrow$ 单元测试。
      其中两个部分完成后进行集成测试,所有部分完成后,进行系统测试。减少错误的风险,同时保证效率。
    \subsection{项目开发的总体计划}
      \subsubsection{软件开发方法}
        参照敏捷开发的思路,整个开发过程用了teambition这个工具实现了任务的发布,团队的交流和任务阶段结束后的总结的目的。
        使得团队能够达到迅速高效的交流沟通。
      \subsubsection{软件需求的处理}
        根据开发软件的实际需求编写文档,让前端和后端工程师根据需求计算工作量。发布每个部分的内容。
        当工程师完成内容后,由测试工程师,做出单元测试,看是否符合软件需求。
      \subsubsection{决策理由的记录}
        通过纪录记录决策。
  \section{详细软件开发活动实施计划}
    \subsection{项目策划和监控}
      目标:用户通过我们提供app软件客户端,可查看是否有人在快递集散地,并通过我们的软件协商实现交易, 消费掉这最后一公里。

      规模:项目尚处于开发阶段,规模较小

      工作量、

      关键计算机资源等估计。

      本条也包括进度的导出方法等。
    \subsection{软件开发环境建立}
      \subsubsection{后端}
        \paragraph{操作系统} Window 7(及以上), Linux(内核4.0版本及以上),OS X 10.9(及以上)
        \paragraph{开发语言} Haskell
        \paragraph{依赖软件} Glasgow Haskell Complier 7.10 (及以上,Windows 要求 7.10.3 以上),Cabal 1.22(及以上),Stack 1.0.0(及以上)
      \subsubsection{数据库}
        \paragraph{数据库管理软件} PostgreSQL
        \paragraph{使用语言} PL/SQL 与 PL/C
     参考后端文档
    \subsubsection{前端}
      \paragraph{操作系统} Windows 10 Build 10586版本(及以上)
      \paragraph{集成开发环境} Visual Studio 2015 Update1(及以上)
      \paragraph{软件开发套件} Windows10 SDK 10586(及以上)
      可参考前端文档
    \subsection{软件需求分析}
      用户注册及用户管理。
      \begin{itemize}
        \item 手机号
        \item 验证码)
        \item 密码
        \item 找回密码
        \item 实名认证
        \item 头像
        \item 昵称
        \item 用户详细的家庭住址
        \item 上传若干张多角度的自行车照片,供其他用户参考
        \item 要求用户选择可以搭载的快递大小和重量
        \item 需要考虑快递是否是易碎物品,若打碎,要有责任承担者 或者不能碰水
      \end{itemize}

      用户通过软件呼叫,实现搭载其他用户自行帮忙携带快递的目的,费用由二者商议决定。
      \begin{itemize}
        \item 利用手机内的gps实现对每个用户的定位,实时计算最适合用户搭载的自行车对象。
        \item 用户通过界面发出请求,系统按优先级(优先级判断)排列出可搭载的自行车对象,返回给界面,让用户做出选择。
        \item 用户和自行车车主可交流商议费用
        \item 利用第三方api实现交易
        \item 发布人信息
        \item 任务状态:已完成 /已接单/未接单
        \item 任务详细内容 A—>B A地址 B地址 任务描述:从A到B 搭自行车或者送快递。送快递要有快递的适当描述 特别要注明是否是易碎物品 不能碰水(不能拆快递)
        \item 快递类别:电子设备 书本 水果 生活用品 信件
        \item 任务注意事项:由用户填写
        \item 交易:代收快递的车主要有实名认证才能接单,单到了确认无误后第三方再把钱打给接单者
        \item 防止垄断:限制每个人接单的次数
        \item 任务流水号
      \end{itemize}

      设置账号等级及评价体系。
      \begin{itemize}
        \item 用户信用跟账号等级直接挂钩,账号等级为优先级判断提供素材
        \item 用户交易完后可对自信车车主进行评价,包括等级评价、文字评价和图片评价,车主可以回复从而进行申诉。其他用户可直接查看该车主的所有评论
      \end{itemize}

      用户间的信息交流

      可参考需求分析文档
    \subsection{软件设计}
      \subsubsection{软件设计任务分布}
        分为前端和后端 分别由前端设计师和后端设计师开发。
      \subsubsection{前后端API调用的实现}
        \paragraph{用户管理}
          \begin{itemize}
             \item 用户注册
             \item 用户认证
             \item 查询认证状态
             \item 用户登入
             \item 用户登出
             \item 查询用户信息
             \item 获取用户的头像
             \item 用户信息修改
             \item 修改密码
             \item 收货地址增删改
          \end{itemize}
        \paragraph{任务管理(代收快递)}
        \begin{itemize}
          \item 任务发布
          \item 任务的删改
          \item 获取任务
          \item 协商价格
          \item 获取任务信息
          \item 支付与确认
          \item 可代收状态的增改
          \item 可代收状态的删除与完成
          \item 可代收状态查询
        \end{itemize}
     \paragraph{消息通告}

     具体实现可参考API文档

   \subsection{软件实现和测试}
     参考单元测试文档,集成测试文档和系统测试文档

  \section{项目组织和资源}
    \subsection{项目组织}
    \subsection{项目资源}
      \subsubsection{人力资源}
        投入人数:4人
        产品经理、前端工程师、后端工程师、测试工程师
      \subsubsection{职责分配}
        \paragraph{产品经理} 负责整个项目的开发计划及软件需求文档的编写。
        \paragraph{前端工程师} 负责前端的设计(包括界面,交互,信息存储等)和前端设计文档的编写。
        \paragraph{后端工程师} 负责后端的设计(包括数据库,业务流程、信息处理等)和后端设计文档的编写。
        \paragraph{测试工程师} 针对每个任务单元测试和测试文档的编写。
    \subsection{项目依赖关系分析}
      产品经理负责发布任务和任务结束时间,并督促工程师,保证任务进度。
      前端和后端工程师要进行定期的讨论,确定任务进度。当一个任务阶段完成后,
      测试工程师应完成单元测试文档,并对当前单元进行测试。以此为基本流程,进行项目开发。

    \subsection{技术方法和工具}
      \paragraph{文档写作} \LaTeX 与 Markdown。
      \paragraph{图形设计} \LaTeX 与 Visio
    \subsection{成本估计}
      \paragraph{时间成本} 开发周期为一个半月
      \paragraph{开发成本} 数据库存储于小型服务器,在docker容器里运行
\end{document}
