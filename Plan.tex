\documentclass[UTF8]{dingo}

\author{康赣鹏 \\ DingoLab \\ganpengkang@gmail.com}
\definecolor{cb}{rgb}{1.00,0.50,0.50}
\definecolor{cf}{rgb}{1,1,1}
\title{Dingo 项目计划书}
\titlebgcolor{cb}
\titlefontcolor{cf}
\coverpic{Dingo-B}
\city{西安}
\version{0.1.0.0}
\begin{document}
  \makecover
  \makecontent
  \section{项目背景}
  随着网购逐渐的渗透于我们的生活,快递也成为我们生活必不可少的一部分。快递方便快捷,为我们的生活减轻了不少负担。然而在许多地区,因为各种各样的原因,快递还是不能实现把商品直达家庭的目标。存在着“最后一公里”,这与快递这种服务诞生的理念相悖。
  \section{国内外市场分析}
  根据我的调查,在中国的大部分高校。因为快递的种类繁杂,人员众多,以及宿舍的集体生活,大部分快递都选择在学校租赁一个快递集散地,让学生自取。这造成一种现象就是:每当下课铃响后,学校的两个地方就会挤满人,一个是饭堂,一个就是快递集散地。将宿舍和快递集散地这段路程利用起来非常有必要。

  还有,在中国的大部分农村,手机通讯,网络的搭建已经开始普及了,但运输快递的网络雏形还没有出现。大部分的村子,想要领取快递必须到附近的镇子上,耗时长还费劲。快递公司若要拓展这部分市场,必须投入大量人力物力,还很难收回成本。这时候,我们可以利用的就是人民自身的力量,填补这部分市场的空缺。

  在国外,许多地广人稀的地方通过类似于快递集散地的驿站,快递的最后一公里还是得客户去消化。这证明,无论是在国内还是国外,“最后一公里”得问题还是没有得到有效得解决,这部分市场的空间很大。

  \section{项目主要开发和建设内容}
    \subsection{项目目标}
    用户通过我们提供app软件客户端,可查看是否有人在快递集散地,并通过我们的软件协商实现交易,消费掉这最后一公里。
    \subsection{开发任务}
    通过的app客户端实现
    \begin{description}
      \item[用户管理] 储存用户信息,用户可通过客户端实现信息管理
      \item[任务管理] 软件主要实现的功能,我们的项目目标
      \item[消息通告] 用户通过这个功能实现人人交互
    \end{description}
  \section{项目实施的技术方案}
    \subsection{具体的任务需求}
      \begin{itemize}
        \item 用户注册及用户管理
        \item 用户通过软件呼叫,实现搭载其他用户自行车帮忙携带快递的目的,费用由二者商议决定
        \item 设置账号等级及评价体系
        \item 用户间的信息交流
        \item 与快递公司和地方商家合作\footnote{更高一阶段的目标}
      \end{itemize}
    \subsection{前后端API调用的具体实现}
      \subsubsection{用户管理}
        \begin{itemize}
          \item 用户注册
          \item 用户认证
          \item 查询认证状态
          \item 用户登入
          \item 用户登出
          \item 查询用户信息
          \item 获取用户的头像
          \item 用户信息修改
          \item 修改密码
          \item 收货地址增删改
        \end{itemize}
      \subsubsection{任务管理(代收快递)}
        \begin{itemize}
          \item 任务发布
          \item 任务的删改
          \item 获取任务
          \item 协商价格
          \item 获取任务信息
          \item 支付与确认
          \item 可代收状态的增改
          \item 可代收状态的删除与完成
          \item 可代收状态查询
        \end{itemize}
      \subsubsection{消息通告}
        代替轮询。
  \section{项目实施情况}
    人员分配,总共有四名,分别担任:
    \begin{itemize}
      \item 项目负责人兼产品经理
      \item 后端工程师
      \item 前端工程师
      \item 软件测试师
    \end{itemize}
    \subsection{项目完成情况}
\end{document}
