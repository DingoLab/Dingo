




%  API.tex


%%%
%%% 项目的 API
%%%


\documentclass[UTF8]{dingo}

%%


\usepackage{datetime}
\usepackage{graphicx}
\usepackage{wrapfig}

% 设置 CJK
\usepackage{xeCJK}
\setCJKmainfont{WenQuanYi Zen Hei}

% 附录
\usepackage[titletoc]{appendix}
\renewcommand\appendixname{附录}
\renewcommand\appendixpagename{附录}
\appendixtitleon

% 宏

\def\apiintr{\paragraph{\colorbox[rgb]{1.0,0.6,0.65}{API 描述}}} % api 的描述
\def\apiexc{\paragraph{\colorbox[rgb]{1,0.85,0.45}{API 输入激励}}} % api 输入的激励
\def\apiresp{\paragraph{\colorbox[rgb]{0.9,0.9,1}{API 输出响应}}} % api 的输出响应
\def\apiauth{\paragraph{\colorbox[rgb]{0.45,0.9,1}{API 身份验证}}} % api 的身份认证
\def\参数名#1{\paragraph{\colorbox[rgb]{0.9,0.5,0.5}{参数名}}#1}
\def\说明{\paragraph{\colorbox[rgb]{0.7,0.7,0.8}{参数说明}}}
\def\范围{\paragraph{\colorbox[rgb]{0.9,0.7,0.7}{参数范围}}}

\def\authT{临时 Token}
\def\authN{无,不验证}

\def\失败{\colorbox[rgb]{1,0.5,0.5}{失败}}
\def\成功{\colorbox[rgb]{0.4,1,0.5}{成功}}
\def\成功V{成功(SUCCESS)}
\def\失败V{失败(FAILED)}
\def\失败原因{失败原因(失败代码,大写字母与数字)}
\def\空{空(NULL)}

% 作者
\author{ 李约瀚 \\ qinka@live.com \\ DingoLab \\ 14130140331 % 排版与记录 后端负责者
    \and 乔新文 \\ starsriver@outlook.com \\ DingoLab \\ 14130140393 % 主要API陈述 UWP 端负责者
    \and 康赣鹏 \\ ganpengkang@gmail.com \\ DingoLab \\ 1413014337 % 铲平经理 负责铲平
    }

%颜色
\definecolor{cb}{rgb}{0,0.55,0.55}
\definecolor{cf}{rgb}{0.9,0.9,0.9}

% 标题
\title{API Document}
\titlebgcolor{cb}
\titlefontcolor{cf}
\coverpic{Dingo-B}
\city{西安}
\version{0.0.2.40}

\begin{document}
    \makecover
    \label{page:contents}
    \pdfbookmark[1]{目录}{page:contents}
    \pagenumbering{roman}
    \setcounter{page}{1}
    \tableofcontents
    \newpage
    \pagenumbering{arabic}
    \setcounter{page}{1}
    %  API 部分
    当前 API 版本 2016-04-16-讨论 \& 个人修改
    \section{用户管理}
    \subsection{用户注册}
    \apiintr
    用于用户注册的 API。同时 用户认证与用户注册并不分在一个API 中。
    \apiexc
    输入:\\
    \begin{tabular}{|c|c|c|c|}
        \hline \rule[-2ex]{0pt}{5.5ex} 参数说明 & 参数名 & 参数范围 & 可选性 \\
        \hline \rule[-2ex]{0pt}{5.5ex} 用户名 & name & 合法的用户名 & 必填 \\
        \hline \rule[-2ex]{0pt}{5.5ex} 密码散列值 & pash & 64位由$0\sim9$、$a\sim f$(小写)组成的散列值 & 必填 \\
        %\hline \rule[-2ex]{0pt}{5.5ex}  &  &  &  \\
        \hline
    \end{tabular}
    \apiresp
    \成功 时返回内容:\\
    \begin{tabular}{|c|c|c|c|}
        \hline \rule[-2ex]{0pt}{5.5ex} 参数名 & 说明 & 值的范围 & 备注 \\
        \hline \rule[-2ex]{0pt}{5.5ex} STATUS & 请求状态 & \成功V &  \\
        \hline \rule[-2ex]{0pt}{5.5ex} CONTEXT & 相关上下文 & 用户的唯一id(数字) &  \\
        %\hline \rule[-2ex]{0pt}{5.5ex}  &  &  &  \\
        \hline
    \end{tabular}
    \par \失败 时返回内容:\\
    \begin{tabular}{|c|c|c|c|}
        \hline \rule[-2ex]{0pt}{5.5ex} 参数名 & 说明 & 值的范围 & 备注 \\
        \hline \rule[-2ex]{0pt}{5.5ex} STATUS & 请求状态 & \失败V &  \\
        \hline \rule[-2ex]{0pt}{5.5ex} CONTEXT & 相关上下文 & \失败原因 &  \\
        %\hline \rule[-2ex]{0pt}{5.5ex}  &  &  &  \\
        \hline
    \end{tabular}
    \apiauth
    不需要任何用户认证。

    \subsection{用户认证}
    \apiintr
    用于用户的认证,提交用户认证的资料。而用户认证的确认这由后台操作数据库。
    同时认证状态有三个:通过认证、未通过认证、待认证。

    电话、邮箱\footnote{目前是选填,之后可能会改变。}、等由客户端与前端检查格式合法性。
    照片是由POST上传文件,目前最大512K。

    上传的照片讲被充当头像,地址并非一定是“收货地址”。
    \apiexc
    输入的:\\
    \begin{tabular}{|c|c|c|c|}
        \hline \rule[-2ex]{0pt}{5.5ex} 参数说明 & 参数名 & 参数范围 & 可选性 \\
        \hline \rule[-2ex]{0pt}{5.5ex} 用户个人手机 & tel & 有效的电话号码 & 必填 \\
        \hline \rule[-2ex]{0pt}{5.5ex} 用户电子邮箱 & email & 合法的电邮地址 & 选填 \\
        \hline \rule[-2ex]{0pt}{5.5ex} 真是姓名 & rname & 合法的中华人民共和国姓名 & 必填 \\
        \hline \rule[-2ex]{0pt}{5.5ex} 身份证 & prcid & 合法的身份证号码 (数字与大写X) & 必填 \\
        \hline \rule[-2ex]{0pt}{5.5ex} 照片 & pic & 二进制数据 & 必填 \\
        \hline \rule[-2ex]{0pt}{5.5ex} 地址 & addr & 文本 & 必填 \\
        %\hline \rule[-2ex]{0pt}{5.5ex}  &  &  &  \\
        \hline
    \end{tabular}
    \apiresp
    \成功 时返回内容:\\
    \begin{tabular}{|c|c|c|c|}
        \hline \rule[-2ex]{0pt}{5.5ex} 参数名 & 说明 & 值的范围 & 备注 \\
        \hline \rule[-2ex]{0pt}{5.5ex} STATUS & 请求状态 & \成功V &  \\
        \hline \rule[-2ex]{0pt}{5.5ex} CONTEXT & 相关上下文 & \空 &  \\
        %\hline \rule[-2ex]{0pt}{5.5ex}  &  &  &  \\
        \hline
    \end{tabular}
    \par \失败 时返回的内容:\\
     \begin{tabular}{|c|c|c|c|}
         \hline \rule[-2ex]{0pt}{5.5ex} 参数名 & 说明 & 值的范围 & 备注 \\
         \hline \rule[-2ex]{0pt}{5.5ex} STATUS & 请求状态 & \失败V &  \\
         \hline \rule[-2ex]{0pt}{5.5ex} CONTEXT & 相关上下文 & \失败原因 &  \\
         %\hline \rule[-2ex]{0pt}{5.5ex}  &  &  &  \\
         \hline
    \end{tabular}
    \apiauth
     需要提供临时的Token

    \subsection{查询认证状态}
    \apiintr
    提供用户查询 API 的状态。
    \apiexc
    无需输入内容\footnote{假定当前仅允许查询自己当前的状态。}。
    \apiresp
    \成功 时返回内容:\\
    \begin{tabular}{|c|c|c|c|}
        \hline \rule[-2ex]{0pt}{5.5ex} 参数名 & 说明 & 值的范围 & 备注 \\
        \hline \rule[-2ex]{0pt}{5.5ex} STATUS & 请求状态 & \成功V &  \\
        \hline \rule[-2ex]{0pt}{5.5ex} CONTEXT & 相关上下文 & 认证状态 &  \\
        %\hline \rule[-2ex]{0pt}{5.5ex}  &  &  &  \\
        \hline
    \end{tabular}
    \par \失败 时返回的内容:\\
    \begin{tabular}{|c|c|c|c|}
        \hline \rule[-2ex]{0pt}{5.5ex} 参数名 & 说明 & 值的范围 & 备注 \\
        \hline \rule[-2ex]{0pt}{5.5ex} STATUS & 请求状态 & \失败V &  \\
        \hline \rule[-2ex]{0pt}{5.5ex} CONTEXT & 相关上下文 & \失败原因 &  \\
        %\hline \rule[-2ex]{0pt}{5.5ex}  &  &  &  \\
        \hline
    \end{tabular}
    \apiauth
    需要提供 临时 Token。

    \subsection{用户登入}
    \apiintr
    用户使用此 API 登陆。使用登陆的优先级为 用户id -> 用户手机 -> 用户名
    \footnote{次设定仅防止用户id、用户名与手机同时传入。}。

    用户的密码使用 特定前缀\footnote{使用用户id 登陆时加上前缀(不包含引号)“uid”,使用用户名则为“nnnn”,电话则为“+86”}
    加上密码的散列值\footnote{使用 SHA256。}
    再加上加上时间戳后再散列\footnote{使用 SHA512。}。
    \apiexc
    当时用用户id登陆时:\\
    \begin{tabular}{|c|c|c|c|}
        \hline \rule[-2ex]{0pt}{5.5ex} 参数说明 & 参数名 & 参数范围 & 可选性 \\
        \hline \rule[-2ex]{0pt}{5.5ex} 用户id & id & 合法的用户名 & 必填 \\
        \hline \rule[-2ex]{0pt}{5.5ex} 用户密码 & pash & 128为的由$0\sim9,a\sim f$组成的字符串 & 必填 \\
        %\hline \rule[-2ex]{0pt}{5.5ex}  &  &  &  \\
        \hline
    \end{tabular}
    \par 当时用用户名登陆时:\\
    \begin{tabular}{|c|c|c|c|}
        \hline \rule[-2ex]{0pt}{5.5ex} 参数说明 & 参数名 & 参数范围 & 可选性 \\
        \hline \rule[-2ex]{0pt}{5.5ex} 用户名 & name & 合法的用户名 & 必填 \\
        \hline \rule[-2ex]{0pt}{5.5ex} 用户密码 & pash & 128为的由$0\sim9,a\sim f$组成的字符串 & 必填 \\
        %\hline \rule[-2ex]{0pt}{5.5ex}  &  &  &  \\
        \hline
    \end{tabular}
    \par 当时用用户手机号登陆时:\\
    \begin{tabular}{|c|c|c|c|}
        \hline \rule[-2ex]{0pt}{5.5ex} 参数说明 & 参数名 & 参数范围 & 可选性 \\
        \hline \rule[-2ex]{0pt}{5.5ex} 用户手机号 & tel & 合法的用户名 & 必填 \\
        \hline \rule[-2ex]{0pt}{5.5ex} 用户密码 & pash & 128为的由$0\sim9,a\sim f$组成的字符串 & 必填 \\
        %\hline \rule[-2ex]{0pt}{5.5ex}  &  &  &  \\
        \hline
    \end{tabular}
    \apiresp
    当登陆 \成功 时的返回内容:\\
    \begin{tabular}{|c|c|c|c|}
        \hline \rule[-2ex]{0pt}{5.5ex} 参数名 & 说明 & 值的范围 & 备注 \\
        \hline \rule[-2ex]{0pt}{5.5ex} STATUS & 请求状态 & \成功V &  \\
        \hline \rule[-2ex]{0pt}{5.5ex} CONTEXT & 相关上下文 & 临时Token &  \\
        %\hline \rule[-2ex]{0pt}{5.5ex}  &  &  &  \\
        \hline
    \end{tabular}
    \par 登陆 \失败 时返回的内容:\\
    \begin{tabular}{|c|c|c|c|}
        \hline \rule[-2ex]{0pt}{5.5ex} 参数名 & 说明 & 值的范围 & 备注 \\
        \hline \rule[-2ex]{0pt}{5.5ex} STATUS & 请求状态 & \失败V &  \\
        \hline \rule[-2ex]{0pt}{5.5ex} CONTEXT & 相关上下文 & \失败原因 &  \\
        %\hline \rule[-2ex]{0pt}{5.5ex}  &  &  &  \\
        \hline
    \end{tabular}
    \apiauth
    使用 密码散列值与用户名认证。
    \subsection{用户登出}
    \apiintr
    用于登出用户的。
    \apiexc 无需任何输入。
    \apiresp
    \成功 时返回的内容:\\
    \\
    \begin{tabular}{|c|c|c|c|}
        \hline \rule[-2ex]{0pt}{5.5ex} 参数名 & 说明 & 值的范围 & 备注 \\
        \hline \rule[-2ex]{0pt}{5.5ex} STATUS & 请求状态 & \成功V &  \\
        \hline \rule[-2ex]{0pt}{5.5ex} CONTEXT & 相关上下文 & \空 &  \\
        %\hline \rule[-2ex]{0pt}{5.5ex}  &  &  &  \\
        \hline
    \end{tabular}
    \par \失败 时返回的内容:\\
    \begin{tabular}{|c|c|c|c|}
        \hline \rule[-2ex]{0pt}{5.5ex} 参数名 & 说明 & 值的范围 & 备注 \\
        \hline \rule[-2ex]{0pt}{5.5ex} STATUS & 请求状态 & \失败V &  \\
        \hline \rule[-2ex]{0pt}{5.5ex} CONTEXT & 相关上下文 & \失败原因 &  \\
        %\hline \rule[-2ex]{0pt}{5.5ex}  &  &  &  \\
        \hline
    \end{tabular}
    \apiauth
    用户的 临时Token。

    \subsection{查询用户信息}
    \apiintr
    用户查询用户信息。
    \apiexc
    输入的内容: \\
    \begin{tabular}{|c|c|c|c|}
        \hline \rule[-2ex]{0pt}{5.5ex} 参数说明 & 参数名 & 参数范围 & 可选性 \\
        \hline \rule[-2ex]{0pt}{5.5ex} 用户id & id & 合法id & 可选 \\
        %\hline \rule[-2ex]{0pt}{5.5ex}  &  &  &  \\
        \hline
    \end{tabular}
    \apiresp
    \成功 时返回的内容:\\
    \\
    \begin{tabular}{|c|c|c|c|}
        \hline \rule[-2ex]{0pt}{5.5ex} 参数名 & 说明 & 值的范围 & 备注 \\
        \hline \rule[-2ex]{0pt}{5.5ex} STATUS & 请求状态 & \成功V &  \\
        \hline \rule[-2ex]{0pt}{5.5ex} CONTEXT & 相关上下文 & 用户信息 &  \\
        %\hline \rule[-2ex]{0pt}{5.5ex}  &  &  &  \\
        \hline
    \end{tabular}
    \par 用户信息包含的内容:\\
    \begin{tabular}{|c|c|c|c|}
        \hline \rule[-2ex]{0pt}{5.5ex} 参数名 & 说明 & 值的范围 & 备注 \\
        \hline \rule[-2ex]{0pt}{5.5ex} id & 用户的id & 合法的id &  \\
        \hline \rule[-2ex]{0pt}{5.5ex} name & 用户的名称 & 合法的用户名称 &  \\
        \hline \rule[-2ex]{0pt}{5.5ex} tel & 用户的手机号 & 合法的电话号码 &  \\
        \hline \rule[-2ex]{0pt}{5.5ex} email & 用户的电邮 & 正常的电邮地址 &  \\
        %\hline \rule[-2ex]{0pt}{5.5ex}  &  &  &  \\
        \hline
    \end{tabular}
    \par \失败 时返回的内容:\\
    \begin{tabular}{|c|c|c|c|}
        \hline \rule[-2ex]{0pt}{5.5ex} 参数名 & 说明 & 值的范围 & 备注 \\
        \hline \rule[-2ex]{0pt}{5.5ex} STATUS & 请求状态 & \失败V &  \\
        \hline \rule[-2ex]{0pt}{5.5ex} CONTEXT & 相关上下文 & \失败原因 &  \\
        %\hline \rule[-2ex]{0pt}{5.5ex}  &  &  &  \\
        \hline
    \end{tabular}
    \apiauth
    需使用临时 Token 认证。
    \subsection{获取用户的头像}
    \apiintr
    用户查询用户的头像。
    \apiexc
    输入的内容: \\
    \begin{tabular}{|c|c|c|c|}
        \hline \rule[-2ex]{0pt}{5.5ex} 参数说明 & 参数名 & 参数范围 & 可选性 \\
        \hline \rule[-2ex]{0pt}{5.5ex} 用户id & id & 合法id & 可选 \\
        %\hline \rule[-2ex]{0pt}{5.5ex}  &  &  &  \\
        \hline
    \end{tabular}
    \apiresp
    \成功 时返回的内容:\\
    \begin{tabular}{|c|c|c|c|}
        \hline \rule[-2ex]{0pt}{5.5ex} 参数名 & 说明 & 值的范围 & 备注 \\
        \hline \rule[-2ex]{0pt}{5.5ex} STATUS & 请求状态 & \成功V &  \\
        \hline \rule[-2ex]{0pt}{5.5ex} CONTEXT & 相关上下文 & 用户头像 & 头像为 body中的数据\\
        %\hline \rule[-2ex]{0pt}{5.5ex}  &  &  &  \\
        \hline
    \end{tabular}
    \par \失败 时返回的内容:\\
    \begin{tabular}{|c|c|c|c|}
        \hline \rule[-2ex]{0pt}{5.5ex} 参数名 & 说明 & 值的范围 & 备注 \\
        \hline \rule[-2ex]{0pt}{5.5ex} STATUS & 请求状态 & \失败V &  \\
        \hline \rule[-2ex]{0pt}{5.5ex} CONTEXT & 相关上下文 & \失败原因 &  \\
        %\hline \rule[-2ex]{0pt}{5.5ex}  &  &  &  \\
        \hline
    \end{tabular}
    \apiauth
    需使用临时 Token 认证。

    \subsection{用户信息修改}
    \apiintr
    用户信息的修改,其中照片为选择性的内容。

    其中当用户的电话,真实姓名,身份证号。
    \apiexc
    输入的信息:\\
    \begin{tabular}{|c|c|c|c|}
        \hline \rule[-2ex]{0pt}{5.5ex} 参数说明 & 参数名 & 参数范围 & 可选性 \\
        \hline \rule[-2ex]{0pt}{5.5ex} 用户名 & name & 合法的用户名 & 必填 \\
        \hline \rule[-2ex]{0pt}{5.5ex} 用户电话 & tel & 合法的电话号码 & 必填 \\
        \hline \rule[-2ex]{0pt}{5.5ex} 用户邮箱 & email & 合法的邮箱 & 必填 \\
        \hline \rule[-2ex]{0pt}{5.5ex} 用户真实姓名 & rname & 合法的姓名 & 必填 \\
        \hline \rule[-2ex]{0pt}{5.5ex} 用户的身份证号 & prcid & 合法,X大写的身份证号码 & 必填 \\
        \hline \rule[-2ex]{0pt}{5.5ex} 用户照片 & pic & 二进制数据 & 可选择 \\
        \hline \rule[-2ex]{0pt}{5.5ex} 用户地址 & addr & 合法的文本 & 必填 \\
        %\hline \rule[-2ex]{0pt}{5.5ex}  &  &  &  \\
        \hline
    \end{tabular}
    \apiresp
    \成功 时返回的内容:\\
    \begin{tabular}{|c|c|c|c|}
        \hline \rule[-2ex]{0pt}{5.5ex} 参数名 & 说明 & 值的范围 & 备注 \\
        \hline \rule[-2ex]{0pt}{5.5ex} STATUS & 请求状态 & \成功V &  \\
        \hline \rule[-2ex]{0pt}{5.5ex} CONTEXT & 相关上下文 & \空 &  \\
        %\hline \rule[-2ex]{0pt}{5.5ex}  &  &  &  \\
        \hline
    \end{tabular}
    \par \失败 时返回的内容:\\
    \begin{tabular}{|c|c|c|c|}
        \hline \rule[-2ex]{0pt}{5.5ex} 参数名 & 说明 & 值的范围 & 备注 \\
        \hline \rule[-2ex]{0pt}{5.5ex} STATUS & 请求状态 & \失败V &  \\
        \hline \rule[-2ex]{0pt}{5.5ex} CONTEXT & 相关上下文 & \失败原因 &  \\
        %\hline \rule[-2ex]{0pt}{5.5ex}  &  &  &  \\
        \hline
    \end{tabular}
    \apiauth
    提供临时Token。

    \subsection{修改密码}
    \apiintr
    提供用户修改用户密码。
    \apiexc
    输入:\\
    \begin{tabular}{|c|c|c|c|}
        \hline \rule[-2ex]{0pt}{5.5ex} 参数说明 & 参数名 & 参数范围 & 可选性 \\
        \hline \rule[-2ex]{0pt}{5.5ex} 密码散列值 & pash & 64位由$0\sim9,a\sim f$(小写)组成的散列值 & 必填 \\
        %\hline \rule[-2ex]{0pt}{5.5ex}  &  &  &  \\
        \hline
    \end{tabular}
    \apiresp
    \成功 时返回的内容:\\
    \begin{tabular}{|c|c|c|c|}
        \hline \rule[-2ex]{0pt}{5.5ex} 参数名 & 说明 & 值的范围 & 备注 \\
        \hline \rule[-2ex]{0pt}{5.5ex} STATUS & 请求状态 & \成功V &  \\
        \hline \rule[-2ex]{0pt}{5.5ex} CONTEXT & 相关上下文 & \空 &  \\
        %\hline \rule[-2ex]{0pt}{5.5ex}  &  &  &  \\
        \hline
    \end{tabular}
    \par \失败 时返回的内容:\\
    \begin{tabular}{|c|c|c|c|}
        \hline \rule[-2ex]{0pt}{5.5ex} 参数名 & 说明 & 值的范围 & 备注 \\
        \hline \rule[-2ex]{0pt}{5.5ex} STATUS & 请求状态 & \失败V &  \\
        \hline \rule[-2ex]{0pt}{5.5ex} CONTEXT & 相关上下文 & \失败原因 &  \\
        %\hline \rule[-2ex]{0pt}{5.5ex}  &  &  &  \\
        \hline
    \end{tabular}
    \apiauth
    使用用户名,密码与临时Token验证。
    \subsection{收货地址的增删改}
    \apiintr
    收货地址的删改。
    \apiexc
    增改的输入:\\
    \begin{tabular}{|c|c|c|c|}
        \hline \rule[-2ex]{0pt}{5.5ex} 参数说明 & 参数名 & 参数范围 & 可选性 \\
        \hline \rule[-2ex]{0pt}{5.5ex} 地址 & addr & 文本 & 必填 \\
        \hline \rule[-2ex]{0pt}{5.5ex} 邮编 & zip & 文本 & 必填 \\
        \hline \rule[-2ex]{0pt}{5.5ex} 地址编号 & aid & 合法的地址编号 & 选填 \\
        \hline \rule[-2ex]{0pt}{5.5ex} 操作 & opt & 区分大小写的文本“AF” & 必填 \\
        %\hline \rule[-2ex]{0pt}{5.5ex}  &  &  &  \\
        \hline
    \end{tabular}
    \par 删除的输入:\\
    \begin{tabular}{|c|c|c|c|}
        \hline \rule[-2ex]{0pt}{5.5ex} 参数说明 & 参数名 & 参数范围 & 可选性 \\
        \hline \rule[-2ex]{0pt}{5.5ex} 地址编号 & aid & 合法的地址编号 & 必填 \\
        \hline \rule[-2ex]{0pt}{5.5ex} 操作 & opt & 区分大小写的文本“DEL” & 必填 \\
        %\hline \rule[-2ex]{0pt}{5.5ex}  &  &  &  \\
        \hline
    \end{tabular}
    \apiresp
    \成功 时返回的内容:\\
    \begin{tabular}{|c|c|c|c|}
        \hline \rule[-2ex]{0pt}{5.5ex} 参数名 & 说明 & 值的范围 & 备注 \\
        \hline \rule[-2ex]{0pt}{5.5ex} STATUS & 请求状态 & \成功V &  \\
        \hline \rule[-2ex]{0pt}{5.5ex} CONTEXT & 相关上下文 & 地址的编号 &  \\
        %\hline \rule[-2ex]{0pt}{5.5ex}  &  &  &  \\
        \hline
    \end{tabular}
    \par \失败 时返回的内容:\\
    \begin{tabular}{|c|c|c|c|}
        \hline \rule[-2ex]{0pt}{5.5ex} 参数名 & 说明 & 值的范围 & 备注 \\
        \hline \rule[-2ex]{0pt}{5.5ex} STATUS & 请求状态 & \失败V &  \\
        \hline \rule[-2ex]{0pt}{5.5ex} CONTEXT & 相关上下文 & \失败原因 &  \\
        %\hline \rule[-2ex]{0pt}{5.5ex}  &  &  &  \\
        \hline
    \end{tabular}
    \apiauth
    使用 临时 Token 与 账户关联度进行验证。


    \subsection{收货地址的查询}
    \apiintr
    收货地址的删改。
    \apiexc
    增改的输入:\\
    \begin{tabular}{|c|c|c|c|}
        \hline \rule[-2ex]{0pt}{5.5ex} 参数说明 & 参数名 & 参数范围 & 可选性 \\
        \hline \rule[-2ex]{0pt}{5.5ex} 用户 id & uid & 文本 & 选填 \\
        %\hline \rule[-2ex]{0pt}{5.5ex}  &  &  &  \\
        \hline
    \end{tabular}
    \apiresp
    \成功 时返回的内容:\\
    \begin{tabular}{|c|c|c|c|}
        \hline \rule[-2ex]{0pt}{5.5ex} 参数名 & 说明 & 值的范围 & 备注 \\
        \hline \rule[-2ex]{0pt}{5.5ex} STATUS & 请求状态 & \成功V &  \\
        \hline \rule[-2ex]{0pt}{5.5ex} CONTEXT & 相关上下文 & 地址信息 &  \\
        %\hline \rule[-2ex]{0pt}{5.5ex}  &  &  &  \\
        \hline
    \end{tabular}
    \par \失败 时返回的内容:\\
    \begin{tabular}{|c|c|c|c|}
        \hline \rule[-2ex]{0pt}{5.5ex} 参数名 & 说明 & 值的范围 & 备注 \\
        \hline \rule[-2ex]{0pt}{5.5ex} STATUS & 请求状态 & \失败V &  \\
        \hline \rule[-2ex]{0pt}{5.5ex} CONTEXT & 相关上下文 & \失败原因 &  \\
        %\hline \rule[-2ex]{0pt}{5.5ex}  &  &  &  \\
        \hline
    \end{tabular}
    \apiauth
    使用 临时 Token 与 账户关联度进行验证。

    \section{任务管理(代收快递)}
    \subsection{任务发布(代收快递)}
    \apiintr
    发布找人代收快递的内容。
    \apiexc
    输入:\\
    \begin{tabular}{|c|c|c|c|}
        \hline \rule[-2ex]{0pt}{5.5ex} 参数说明 & 参数名 & 参数范围 & 可选性 \\
        \hline \rule[-2ex]{0pt}{5.5ex} 经度 & ew & $[-180^\circ,180^\circ]$ 其中正向为东 & 必填 \\
        \hline \rule[-2ex]{0pt}{5.5ex} 纬度 & ns & $[-90^\circ,90^\circ]$ & 必填 \\
        \hline \rule[-2ex]{0pt}{5.5ex} 范围 & r & 有效以米为单位的浮点数 & 必填 \\
        \hline \rule[-2ex]{0pt}{5.5ex} 快递的重量 & wei & 以千克为单位的浮点数 & 必填 \\
        \hline \rule[-2ex]{0pt}{5.5ex} 快递的尺寸 & size & 三元组 $(x,y,z)$ 表示长宽高 & 必填 \\
        \hline \rule[-2ex]{0pt}{5.5ex} 注意事项 & note & 文本 & 选填 \\
        \hline \rule[-2ex]{0pt}{5.5ex} 价格 & cost & 以人民币中分为单位的整数 & 必填 \\
        \hline \rule[-2ex]{0pt}{5.5ex} 描述 & intr & 文本 & 选填 \\
        \hline \rule[-2ex]{0pt}{5.5ex} 关联的乙方 & linked & 有效的用户id & 选填 \\
        %\hline \rule[-2ex]{0pt}{5.5ex}  &  &  &  \\
        \hline
    \end{tabular}
    \apiresp
    \成功 时返回的内容:\\
    \begin{tabular}{|c|c|c|c|}
        \hline \rule[-2ex]{0pt}{5.5ex} 参数名 & 说明 & 值的范围 & 备注 \\
        \hline \rule[-2ex]{0pt}{5.5ex} STATUS & 请求状态 & \成功V &  \\
        \hline \rule[-2ex]{0pt}{5.5ex} CONTEXT & 相关上下文 & 唯一的任务编号 &  \\
        %\hline \rule[-2ex]{0pt}{5.5ex}  &  &  &  \\
        \hline
    \end{tabular}
    \par \失败 时返回的内容:\\
    \begin{tabular}{|c|c|c|c|}
        \hline \rule[-2ex]{0pt}{5.5ex} 参数名 & 说明 & 值的范围 & 备注 \\
        \hline \rule[-2ex]{0pt}{5.5ex} STATUS & 请求状态 & \失败V &  \\
        \hline \rule[-2ex]{0pt}{5.5ex} CONTEXT & 相关上下文 & \失败原因 &  \\
        %\hline \rule[-2ex]{0pt}{5.5ex}  &  &  &  \\
        \hline
    \end{tabular}
    \apiauth
    使用 临时Token 认证。
    \subsection{任务的删改}
    \apiintr
    用户删除或者同时修改。
    \apiexc 当删除操作时:\\
    \begin{tabular}{|c|c|c|c|}
        \hline \rule[-2ex]{0pt}{5.5ex} 参数说明 & 参数名 & 参数范围 & 可选性 \\
        \hline \rule[-2ex]{0pt}{5.5ex} 操作类型 & ct & DEL & 必填 \\
        \hline \rule[-2ex]{0pt}{5.5ex} 任务id & tid & 合法的任务id & 必填 \\
        %\hline \rule[-2ex]{0pt}{5.5ex}  &  &  &  \\
        \hline
    \end{tabular}

    \par 当执行修改操作: \\
    \begin{tabular}{|c|c|c|c|}
        \hline \rule[-2ex]{0pt}{5.5ex} 参数说明 & 参数名 & 参数范围 & 可选性 \\
        \hline \rule[-2ex]{0pt}{5.5ex} 操作类型 & ct & FIX & 必填 \\
        \hline \rule[-2ex]{0pt}{5.5ex} 任务id & tid & 合法的任务id & 必填 \\
        \hline \rule[-2ex]{0pt}{5.5ex} 经度 & ew & $[-180^\circ,180^\circ]$ 其中正向为东 & 必填 \\
        \hline \rule[-2ex]{0pt}{5.5ex} 纬度 & ns & $[-90^\circ,90^\circ]$ & 必填 \\
        \hline \rule[-2ex]{0pt}{5.5ex} 范围 & r & 有效以米为单位的浮点数 & 必填 \\
        \hline \rule[-2ex]{0pt}{5.5ex} 快递的重量 & wei & 以千克为单位的浮点数 & 必填 \\
        \hline \rule[-2ex]{0pt}{5.5ex} 快递的尺寸 & size & 三元组 $(x,y,z)$ 表示长宽高 & 必填 \\
        \hline \rule[-2ex]{0pt}{5.5ex} 注意事项 & note & 文本 & 选填 \\
        \hline \rule[-2ex]{0pt}{5.5ex} 价格 & cost & 以人民币中分为单位的整数 & 必填 \\
        \hline \rule[-2ex]{0pt}{5.5ex} 描述 & intr & 文本 & 选填 \\
        \hline \rule[-2ex]{0pt}{5.5ex} 关联的乙方 & linked & 有效的用户id & 选填 \\
        %\hline \rule[-2ex]{0pt}{5.5ex}  &  &  &  \\
        \hline
    \end{tabular}
    \apiresp
    \成功 时返回的内容:\\
    \begin{tabular}{|c|c|c|c|}
        \hline \rule[-2ex]{0pt}{5.5ex} 参数名 & 说明 & 值的范围 & 备注 \\
        \hline \rule[-2ex]{0pt}{5.5ex} STATUS & 请求状态 & \成功V &  \\
        \hline \rule[-2ex]{0pt}{5.5ex} CONTEXT & 相关上下文 & \空 &  \\
        %\hline \rule[-2ex]{0pt}{5.5ex}  &  &  &  \\
        \hline
    \end{tabular}
    \par \失败 时返回的内容:\\
    \begin{tabular}{|c|c|c|c|}
        \hline \rule[-2ex]{0pt}{5.5ex} 参数名 & 说明 & 值的范围 & 备注 \\
        \hline \rule[-2ex]{0pt}{5.5ex} STATUS & 请求状态 & \失败V &  \\
        \hline \rule[-2ex]{0pt}{5.5ex} CONTEXT & 相关上下文 & \失败原因 &  \\
        %\hline \rule[-2ex]{0pt}{5.5ex}  &  &  &  \\
        \hline
    \end{tabular}
    \apiauth
    使用 临时Token 与 账户关联\footnote{是否为甲乙方,无关的内容视为无访问权限。}度验证。
    \subsection{获取任务(代收快递)}
    \apiexc 输入: \\
    \begin{tabular}{|c|c|c|c|}
        \hline \rule[-2ex]{0pt}{5.5ex} 参数说明 & 参数名 & 参数范围 & 可选性 \\
        \hline \rule[-2ex]{0pt}{5.5ex} 经度 & ew & $[-180^\circ,180^\circ]$ 其中正向为东 & 必填 \\
        \hline \rule[-2ex]{0pt}{5.5ex} 纬度 & ns & $[-90^\circ,90^\circ]$ & 必填 \\
        \hline \rule[-2ex]{0pt}{5.5ex} 范围 & r & 有效以米为单位的浮点数 & 必填 \\
        %\hline \rule[-2ex]{0pt}{5.5ex}  &  &  &  \\
        \hline
    \end{tabular}
    \apiresp
    \成功 时返回的内容:\\
    \begin{tabular}{|c|c|c|c|}
        \hline \rule[-2ex]{0pt}{5.5ex} 参数名 & 说明 & 值的范围 & 备注 \\
        \hline \rule[-2ex]{0pt}{5.5ex} STATUS & 请求状态 & \成功V &  \\
        \hline \rule[-2ex]{0pt}{5.5ex} CONTEXT & 相关上下文 & 返回数据 &  \\
        %\hline \rule[-2ex]{0pt}{5.5ex}  &  &  &  \\
        \hline
    \end{tabular}
    \par 返回数据: \\
    \begin{tabular}{|c|c|c|c|}
        \hline \rule[-2ex]{0pt}{5.5ex} 参数名 & 说明 & 值的范围 & 备注 \\
        \hline \rule[-2ex]{0pt}{5.5ex} count & 返回的任务编号的数量 & 正整数 &  \\
        \hline \rule[-2ex]{0pt}{5.5ex} data & 返回的任务编号 & 数组列表 &  \\
        %\hline \rule[-2ex]{0pt}{5.5ex}  &  &  &  \\
        \hline
    \end{tabular}
    \par \失败 时返回的内容:\\
    \begin{tabular}{|c|c|c|c|}
        \hline \rule[-2ex]{0pt}{5.5ex} 参数名 & 说明 & 值的范围 & 备注 \\
        \hline \rule[-2ex]{0pt}{5.5ex} STATUS & 请求状态 & \失败V &  \\
        \hline \rule[-2ex]{0pt}{5.5ex} CONTEXT & 相关上下文 & \失败原因 &  \\
        %\hline \rule[-2ex]{0pt}{5.5ex}  &  &  &  \\
        \hline
    \end{tabular}
    \apiauth
    使用 临时Token 与 账户关联度验证。
    \subsection{协商价格}
    \apiintr
    用于协商价格。
    \apiexc
    价格修改\footnote{并不修改价格,只是在实际价格上做加法,会有专门的数据记录价格变动。}的输入:\\
    \begin{tabular}{|c|c|c|c|}
        \hline \rule[-2ex]{0pt}{5.5ex} 参数说明 & 参数名 & 参数范围 & 可选性 \\
        \hline \rule[-2ex]{0pt}{5.5ex} 任务id & tid & 合法的id & 必填 \\
        \hline \rule[-2ex]{0pt}{5.5ex} 调整的价格 & d & 正负的以分为单位的整数 & 必填 \\
        \hline \rule[-2ex]{0pt}{5.5ex} 是否继续 & c & 字符串 “True” 首字母大写 & 必填 \\
        %\hline \rule[-2ex]{0pt}{5.5ex}  &  &  &  \\
        \hline
    \end{tabular}
    \par 放弃当前“交易/协商”,停止议价并回到最初状态: \\
    \begin{tabular}{|c|c|c|c|}
        \hline \rule[-2ex]{0pt}{5.5ex} 参数说明 & 参数名 & 参数范围 & 可选性 \\
        \hline \rule[-2ex]{0pt}{5.5ex} 任务id & tid & 合法的id & 必填 \\
        \hline \rule[-2ex]{0pt}{5.5ex} 是否继续 & c & 字符串 “False” 首字母大写 & 必填 \\
        %\hline \rule[-2ex]{0pt}{5.5ex}  &  &  &  \\
        \hline
    \end{tabular}
    \apiresp
    \成功 时返回的内容:\\
    \begin{tabular}{|c|c|c|c|}
        \hline \rule[-2ex]{0pt}{5.5ex} 参数名 & 说明 & 值的范围 & 备注 \\
        \hline \rule[-2ex]{0pt}{5.5ex} STATUS & 请求状态 & \成功V &  \\
        \hline \rule[-2ex]{0pt}{5.5ex} CONTEXT & 相关上下文 & \空 &  \\
        %\hline \rule[-2ex]{0pt}{5.5ex}  &  &  &  \\
        \hline
    \end{tabular}
    \par \失败 时返回的内容:\\
    \begin{tabular}{|c|c|c|c|}
        \hline \rule[-2ex]{0pt}{5.5ex} 参数名 & 说明 & 值的范围 & 备注 \\
        \hline \rule[-2ex]{0pt}{5.5ex} STATUS & 请求状态 & \失败V &  \\
        \hline \rule[-2ex]{0pt}{5.5ex} CONTEXT & 相关上下文 & \失败原因 &  \\
        %\hline \rule[-2ex]{0pt}{5.5ex}  &  &  &  \\
        \hline
    \end{tabular}
    \apiauth
    使用 临时 Token 与账户关联度认证。
    \subsection{获取任务信息}
    \apiintr
    用于获取任务信息。
    \apiexc
    输入:\\
    \begin{tabular}{|c|c|c|c|}
        \hline \rule[-2ex]{0pt}{5.5ex} 参数说明 & 参数名 & 参数范围 & 可选性 \\
        \hline \rule[-2ex]{0pt}{5.5ex} 任务id & id & 合法的任务id & 必填 \\
        %\hline \rule[-2ex]{0pt}{5.5ex}  &  &  &  \\
        \hline
    \end{tabular}
    \apiresp
    \成功 时返回的内容:\\
    \begin{tabular}{|c|c|c|c|}
        \hline \rule[-2ex]{0pt}{5.5ex} 参数名 & 说明 & 值的范围 & 备注 \\
        \hline \rule[-2ex]{0pt}{5.5ex} STATUS & 请求状态 & \成功V &  \\
        \hline \rule[-2ex]{0pt}{5.5ex} CONTEXT & 相关上下文 & 任务相关信息 &  \\
        %\hline \rule[-2ex]{0pt}{5.5ex}  &  &  &  \\
        \hline
    \end{tabular}
    \par 任务相关信息: \\
    \begin{tabular}{|c|c|c|c|}
        \hline \rule[-2ex]{0pt}{5.5ex} 参数说明 & 参数名 & 参数范围 & 备注 \\        \hline \rule[-2ex]{0pt}{5.5ex} 操作类型 & ct & FIX & 必填 \\
        \hline \rule[-2ex]{0pt}{5.5ex} 任务id & tid & 合法的任务id &  \\
        \hline \rule[-2ex]{0pt}{5.5ex} 经度 & ew & $[-180^\circ,180^\circ]$ 其中正向为东 &  \\
        \hline \rule[-2ex]{0pt}{5.5ex} 纬度 & ns & $[-90^\circ,90^\circ]$ &  \\
        \hline \rule[-2ex]{0pt}{5.5ex} 范围 & r & 有效以米为单位的浮点数 &  \\
        \hline \rule[-2ex]{0pt}{5.5ex} 快递的重量 & wei & 以千克为单位的浮点数 &  \\
        \hline \rule[-2ex]{0pt}{5.5ex} 快递的尺寸 & size & 三元组 $(x,y,z)$ 表示长宽高 &  \\
        \hline \rule[-2ex]{0pt}{5.5ex} 注意事项 & note & 文本 &  \\
        \hline \rule[-2ex]{0pt}{5.5ex} 价格 & cost & 以人民币中分为单位的整数 &  \\
        \hline \rule[-2ex]{0pt}{5.5ex} 甲方价格修正 & costA & 以人民币中分为单位的整数的数组 &  \\
        \hline \rule[-2ex]{0pt}{5.5ex} 乙方价格修正 & costB & 以人民币中分为单位的整数的数组 &  \\
        \hline \rule[-2ex]{0pt}{5.5ex} 描述 & intr & 文本 & 选填 \\
        \hline \rule[-2ex]{0pt}{5.5ex} 关联的甲方 & linkedA & 有效的用户id &  \\
        \hline \rule[-2ex]{0pt}{5.5ex} 关联的乙方 & linkedB & 有效的用户id &  \\
        %\hline \rule[-2ex]{0pt}{5.5ex}  &  &  &  \\
        \hline
    \end{tabular}
    \par \失败 时返回的内容:\\
    \begin{tabular}{|c|c|c|c|}
        \hline \rule[-2ex]{0pt}{5.5ex} 参数名 & 说明 & 值的范围 & 备注 \\
        \hline \rule[-2ex]{0pt}{5.5ex} STATUS & 请求状态 & \失败V &  \\
        \hline \rule[-2ex]{0pt}{5.5ex} CONTEXT & 相关上下文 & \失败原因 &  \\
        %\hline \rule[-2ex]{0pt}{5.5ex}  &  &  &  \\
        \hline
    \end{tabular}
    \apiauth
    使用 临时Token 与 账号关联度 验证
    \subsection{支付与确认}
    \apiintr
    用户双方的支付与确认。
    \apiexc
    输入信息:\\
    \begin{tabular}{|c|c|c|c|}
        \hline \rule[-2ex]{0pt}{5.5ex} 参数说明 & 参数名 & 参数范围 & 可选性 \\
        \hline \rule[-2ex]{0pt}{5.5ex} 任务编号 & tid & 合法的任务编号 & 必填 \\
        %\hline \rule[-2ex]{0pt}{5.5ex}  &  &  &  \\
        \hline
    \end{tabular}
    \apiresp
    \成功 时返回的内容:\\
    \begin{tabular}{|c|c|c|c|}
        \hline \rule[-2ex]{0pt}{5.5ex} 参数名 & 说明 & 值的范围 & 备注 \\
        \hline \rule[-2ex]{0pt}{5.5ex} STATUS & 请求状态 & \成功V &  \\
        \hline \rule[-2ex]{0pt}{5.5ex} CONTEXT & 相关上下文 & \空 &  \\
        %\hline \rule[-2ex]{0pt}{5.5ex}  &  &  &  \\
        \hline
    \end{tabular}
    \par \失败 时返回的内容:\\
    \begin{tabular}{|c|c|c|c|}
        \hline \rule[-2ex]{0pt}{5.5ex} 参数名 & 说明 & 值的范围 & 备注 \\
        \hline \rule[-2ex]{0pt}{5.5ex} STATUS & 请求状态 & \失败V &  \\
        \hline \rule[-2ex]{0pt}{5.5ex} CONTEXT & 相关上下文 & \失败原因 &  \\
        %\hline \rule[-2ex]{0pt}{5.5ex}  &  &  &  \\
        \hline
    \end{tabular}
    \apiauth
    使用 临时Token 与 账户关联度 来验证。
    \subsection{可代收状态的增改}
    \apiintr
    用于增加与修改“可代收状态”。
    \apiexc
    输入: \\
    \begin{tabular}{|c|c|c|c|}
        \hline \rule[-2ex]{0pt}{5.5ex} 参数说明 & 参数名 & 参数范围 & 可选性 \\
        \hline \rule[-2ex]{0pt}{5.5ex} 代收任务编号 & did & 合法的代收任务编号 & 选填,若无则创建新的代收 \\
        \hline \rule[-2ex]{0pt}{5.5ex} 代收描述 & dd & 文本 & 必填 \\
        \hline \rule[-2ex]{0pt}{5.5ex} 经度 & ew & $[-180^\circ,180^\circ]$ 其中正向为东 & 必填 \\
        \hline \rule[-2ex]{0pt}{5.5ex} 纬度 & ns & $[-90^\circ,90^\circ]$ & 必填 \\
        \hline \rule[-2ex]{0pt}{5.5ex} 范围 & r & 有效以米为单位的浮点数 & 必填 \\
        %\hline \rule[-2ex]{0pt}{5.5ex}  &  &  &  \\
        \hline
    \end{tabular}
    \apiresp
    \成功 时返回的内容:\\
    \begin{tabular}{|c|c|c|c|}
        \hline \rule[-2ex]{0pt}{5.5ex} 参数名 & 说明 & 值的范围 & 备注 \\
        \hline \rule[-2ex]{0pt}{5.5ex} STATUS & 请求状态 & \成功V &  \\
        \hline \rule[-2ex]{0pt}{5.5ex} CONTEXT & 相关上下文 & 代收状态编号 &  \\
        %\hline \rule[-2ex]{0pt}{5.5ex}  &  &  &  \\
        \hline
    \end{tabular}
    \par \失败 时返回的内容:\\
    \begin{tabular}{|c|c|c|c|}
        \hline \rule[-2ex]{0pt}{5.5ex} 参数名 & 说明 & 值的范围 & 备注 \\
        \hline \rule[-2ex]{0pt}{5.5ex} STATUS & 请求状态 & \失败V &  \\
        \hline \rule[-2ex]{0pt}{5.5ex} CONTEXT & 相关上下文 & \失败原因 &  \\
        %\hline \rule[-2ex]{0pt}{5.5ex}  &  &  &  \\
        \hline
    \end{tabular}
    \apiauth
    临时 Token 验证。
    \subsection{可代收状态的删与完}
    \apiintr
    用于可查询状态的删除与“执行”。
    \apiexc
    删除时的输入:\\
    \begin{tabular}{|c|c|c|c|}
        \hline \rule[-2ex]{0pt}{5.5ex} 参数说明 & 参数名 & 参数范围 & 可选性 \\
        \hline \rule[-2ex]{0pt}{5.5ex} 删除的代收状态号 & did & 合法的代收状态号 & 必填 \\
        \hline \rule[-2ex]{0pt}{5.5ex} 操作 & opt & 区分大小写的字符串“DEL” & 必填 \\
        %\hline \rule[-2ex]{0pt}{5.5ex}  &  &  &  \\
        \hline
    \end{tabular}
    \par 完成的输入: \\
    \begin{tabular}{|c|c|c|c|}
        \hline \rule[-2ex]{0pt}{5.5ex} 参数说明 & 参数名 & 参数范围 & 可选性 \\
        \hline \rule[-2ex]{0pt}{5.5ex} 删除的代收状态号 & did & 合法的代收状态号 & 必填 \\
        \hline \rule[-2ex]{0pt}{5.5ex} 操作 & opt & 区分大小写的字符串“FIN” & 必填 \\
        %\hline \rule[-2ex]{0pt}{5.5ex}  &  &  &  \\
        \hline
    \end{tabular}
    \apiresp
    返回的内容:\\
    \成功 时返回的内容:\\
    \begin{tabular}{|c|c|c|c|}
        \hline \rule[-2ex]{0pt}{5.5ex} 参数名 & 说明 & 值的范围 & 备注 \\
        \hline \rule[-2ex]{0pt}{5.5ex} STATUS & 请求状态 & \成功V &  \\
        \hline \rule[-2ex]{0pt}{5.5ex} CONTEXT & 相关上下文 & \空 &  \\
        %\hline \rule[-2ex]{0pt}{5.5ex}  &  &  &  \\
        \hline
    \end{tabular}
    \par \失败 时返回的内容:\\
    \begin{tabular}{|c|c|c|c|}
        \hline \rule[-2ex]{0pt}{5.5ex} 参数名 & 说明 & 值的范围 & 备注 \\
        \hline \rule[-2ex]{0pt}{5.5ex} STATUS & 请求状态 & \失败V &  \\
        \hline \rule[-2ex]{0pt}{5.5ex} CONTEXT & 相关上下文 & \失败原因 &  \\
        %\hline \rule[-2ex]{0pt}{5.5ex}  &  &  &  \\
        \hline
    \end{tabular}
    \apiauth
    使用 临时Token 与账号相关度验证。
    \subsection{克代收状态查询}
    \apiintr
    查询当前的可代收双胎。
    \apiexc
    \begin{tabular}{|c|c|c|c|}
        \hline \rule[-2ex]{0pt}{5.5ex} 参数说明 & 参数名 & 参数范围 & 可选性 \\
        \hline \rule[-2ex]{0pt}{5.5ex} 经度 & ew & $[-180^\circ,180^\circ]$ 其中正向为东 & 必填 \\
        \hline \rule[-2ex]{0pt}{5.5ex} 纬度 & ns & $[-90^\circ,90^\circ]$ 其中以北为正向 & 必填 \\
        \hline \rule[-2ex]{0pt}{5.5ex} 范围 & r & 有效以米为单位的浮点数 & 必填 \\
        %\hline \rule[-2ex]{0pt}{5.5ex}  &  &  &  \\
        \hline
    \end{tabular}
    \apiresp
    返回的内容:\\
    \成功 时返回的内容:\\
    \begin{tabular}{|c|c|c|c|}
        \hline \rule[-2ex]{0pt}{5.5ex} 参数名 & 说明 & 值的范围 & 备注 \\
        \hline \rule[-2ex]{0pt}{5.5ex} STATUS & 请求状态 & \成功V &  \\
        \hline \rule[-2ex]{0pt}{5.5ex} CONTEXT & 相关上下文 & 代收信息 &  \\
        %\hline \rule[-2ex]{0pt}{5.5ex}  &  &  &  \\
        \hline
    \end{tabular}
    \par 代收信息: \\
    \begin{tabular}{|c|c|c|c|}
        \hline \rule[-2ex]{0pt}{5.5ex} 参数名 & 说明 & 值的范围 & 备注 \\
        \hline \rule[-2ex]{0pt}{5.5ex} dd & 代收描述 &  &  \\
        \hline \rule[-2ex]{0pt}{5.5ex} ew & 经度 & $[-180^\circ,180^\circ]$ 其中正向为东 &  \\
        \hline \rule[-2ex]{0pt}{5.5ex} ns & 纬度 & $[-90^\circ,90^\circ]$ 其中以北为正向 &  \\
        \hline \rule[-2ex]{0pt}{5.5ex} r & 范围 & 有效以米为单位的浮点数 &  \\
        %\hline \rule[-2ex]{0pt}{5.5ex}  &  &  &  \\
        \hline
    \end{tabular}
    \par \失败 时返回的内容:\\
    \begin{tabular}{|c|c|c|c|}
        \hline \rule[-2ex]{0pt}{5.5ex} 参数名 & 说明 & 值的范围 & 备注 \\
        \hline \rule[-2ex]{0pt}{5.5ex} STATUS & 请求状态 & \失败V &  \\
        \hline \rule[-2ex]{0pt}{5.5ex} CONTEXT & 相关上下文 & \失败原因 &  \\
        %\hline \rule[-2ex]{0pt}{5.5ex}  &  &  &  \\
        \hline
    \end{tabular}
    \apiauth
    通过 临时Token 验证。
    \section{消息通告}
    \subsection{消息通告}
    使用 TCP 链接。
    \newpage
    \begin{appendices}
        \section{常用变量说明}
        \subsection{用户名}
            \参数名{name}
            \说明 用于标识用户的标识符之一。
            \范围 合法的字符串,包括 大小写的英文字母、数字\footnote{不用于开头}与下划线。
        \subsection{密码散列值(注册用)}
            \参数名{pash}
            \说明 注册用的密码散列值,使用的是 SHA256 算法,不包含时间戳。登陆用的包含时间戳,
            并使用 SHA384 算法。登陆时特定前缀,密码散列值与时间戳拼接后在进行散列。
            \范围 64位或者96位长度的字符串,由数字$0\sim9$与部分小写字母$a\sim f$组成的
            十六进制表示散列值。
        \subsection{返回状态(STATUS)}
            \参数名{STATUS}
            \说明 确定每个HTTP请求的响应状态。其中一般成功想要的返回的状态的编码
            \footnote{这里的编码是指HTTP响应的状态编码。}是 200。'
            当返回的状态不是成功的,则返回的是 4xxx 四位的状态码
            \footnote{此处与一般的HTTP的状态码不同,此处为自定义的 4位的码}
            或者是标准的三位的状态码。其中具体的返回内容组织形式,需要阅读 附录B。
            \范围 一共两种状态, “SUCCESS” 与 “FAIL” 。前者表示成功,后者笼统地表示错误,
            具体错误需要检查相关上下文中返回的数据。
        \subsection{相关上下文(CONTEXT)}
            \参数名{CONTEXT}
            \说明 返回内容中核心的数据——相关上下文。细节请参考 附录B 中的内容。
            \范围 “任何” 合法的数据。
        \subsection{相关上下文位置}
            \参数名{CONTEXT-WHERE}
            \说明 指明相关上下文的位置。
            \范围 具体细节需要参考 附录B 中的内容。
        \subsection{用户电话}
            \参数名{tel}
            \说明 用户的手机号码。
            \范围 合法的手机号码。
            以 1开头的 十一位手机号码。无需“+8”6前缀
        \subsection{用户的电邮}
            \参数名{email}
            \说明 用户使用的电子邮箱地址。
            \范围 用户名@邮箱域名
        \subsection{真实姓名}
            \参数名{rname}
            \说明 用户在中华人民共和国户籍系统或者其他户籍单位注册的当前使用的用户姓名。
            \范围 一定长度的文本。
        \subsection{身份证件号}
            \参数名{prcid}
            \说明 在中华人民共和国户籍系统中,表示每个公民唯一性的那个身份证号。
            \范围 18位数字,但其中最后一位是可以为X的(大写)。
        \subsection{地址}
            \参数名{addr\footnote{可能某些情况并不是这个名称。}}
            \说明 是指代收货地址或者其它相关地址。
            \范围 文本。
        \subsection{用户id}
            \参数名{id}
            \说明 是每个用户在系统中的唯一的标识码
            \范围 以字母 U 开头,并以日期与一串散列值\footnote{长度为40个字符。}组成。其中 日期的格式是:yyyy-mm-dd。
            字母,日期与散列值。
        \subsection{临时 Token}
            \参数名{TMP-TOKEN}
            \说明 用于认证的一串字符串,其位于 HTTP 请求的头中。
            \范围 一串带有时间戳的,长度为 150 个字节的字符串。
        \subsection{地址内容:省}
            \参数名{p}
            \说明 合法的中华人民共和国省级行政区域。
            \范围 中华人民共和国的直辖市、省、自治区、特区。
        \subsection{地址内容:地}
            \参数名{c}
            \说明 合法的中华人民共和国地级行政区域。
            \范围 仅次于省级的一级行政区域,包含地级市、地区、自治州、盟。
        \subsection{地址内容:县}
            \参数名{d}
            \说明 合法的中华人民共和国县级行政区域。
            \范围 仅次于地级的一级行政区域,包括市辖区、县级市、
            县、自治县、旗、自治旗、特区、林区。
        \subsection{地址内容:乡}
            \参数名{b}
            \说明 合法的中华人民共和国乡级行政区域。
            \范围 对于乡,为仅次于县级的一级行政区域,包括区公所、街道、镇、
            乡、民族乡、苏木、民族苏木。
        \subsection{地址内容:村或者小区}
            \参数名{v}
            \说明 合法的中华人民共和国村级行政区域与从城市中的小区、
            大型单位的场地、校区机构等。
            \范围 仅次于乡级的行政区域或其他。
        \subsection{地址内容:邮编}
            \参数名{zip}
            \说明 中华人民共和国内的邮政编码。
            \范围 6位有效的邮政编码。
        \subsection{地址编号}
            \参数名{aid}
            \说明 每个地址有一个唯一的编号。
            \范围 以 A 开头,之后是日期\footnote{同用户di中的日期。},
            然后是地址的SHA256 散列。
        \subsection{坐标信息:经度}
            \参数名{ew}
            \说明 划分地球的经度。
            \范围 $[-180^\circ,+180^\circ]$,其中,正方向为东。
        \subsection{坐标信息:纬度}
            \参数名{ns}
            \说明 划分地球的纬度。
            \范围 $[-90^\circ,+90^\circ]$,其中,正方向为北。
        \subsection{范围}
            \参数名{r}
            \说明 划分一个范围。
            \范围 以米为单位的浮点数。
        \subsection{快递的重量}
            \参数名{wei}
            \说明 说明快递的重量
            \范围 以千克为单位的浮点数。
        \subsection{快递的尺寸}
            \参数名{size}
            \说明 说明快递的尺寸。
            \范围 以厘米为单位的三元组 $(x,y,z)$ 分别表示 长宽高。
        \subsection{价格}
            \参数名{cost}
            \说明 说明当前交易的价格。
            \范围 以人民币中分为单位的整数。
        \subsection{任务id}
            \参数名{tid}
            \说明 每个任务的唯一标识符。
            \范围 以 T 开头,后面加上日期\footnote{与用户id相同。},再加上一个长度为
            40的散列值。
        %\subsection{}
        %    \参数名{}
        %    \说明
        %    \范围
        %\subsection{}
        %    \参数名{}
        %    \说明
        %    \范围
        %\subsection{}
        %    \参数名{}
        %    \说明
        %    \范围
        %\subsection{}
        %    \参数名{}
        %    \说明
        %    \范围
        %\subsection{}
        %    \参数名{}
        %    \说明
        %    \范围
        %\subsection{}
        %    \参数名{}
        %    \说明
        %    \范围
        %\subsection{}
        %    \参数名{}
        %    \说明
        %    \范围
        \section{返回数据的实际组织方式}
        实际的数据组织方式相对较为复杂。首先返回的HTTP报文中头部将会有一个字段——“STATUS”,
        这个字段中会有两种取值 “SUCCESS” 与 “FAILED” 分别代表成功与失败。然后会有一个字段——
        “CONTEXT-WHERE”,这个字段表明了返回数据的位置,取值或为“BODY”,或为“HEAD”,或为“OTH”
        。通常需要API实现约定好,同时有时会出现同时出现在多个地方的情况。
    \end{appendices}

\end{document}

%\subsection{}
%    \参数名{}
%    \说明
%    \范围

%\begin{tabular}{|c|c|c|c|}
%    \hline \rule[-2ex]{0pt}{5.5ex} 参数说明 & 参数名 & 参数范围 & 可选性 \\
%    %\hline \rule[-2ex]{0pt}{5.5ex}  &  &  &  \\
%    \hline
%\end{tabular}

%\begin{tabular}{|c|c|c|c|}
%    \hline \rule[-2ex]{0pt}{5.5ex} 参数名 & 说明 & 值的范围 & 备注 \\
%    %\hline \rule[-2ex]{0pt}{5.5ex}  &  &  &  \\
%    \hline
%\end{tabular}
